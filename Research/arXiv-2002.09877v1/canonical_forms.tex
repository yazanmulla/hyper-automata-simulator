
\subsection{Canonical Forms for the Alternation-Free Fragments}\label{subsec:canonical.forms}

%Throughout this section, we discuss NFH over $\Sigma$ and a set of variables $X$.
We begin with the basic terms on which our canonical forms are based.
\begin{definition}
\begin{enumerate}
\item An $\nfhf$ $\A_\forall$ is {\em sequence complete} if for every word ${\bi w}$, it holds that $\hat{\A_\forall}$ accepts ${\bi w}$ iff it accepts every sequence of ${\bi w}$. 
\item An $\nfhe$ $\A_\exists$ is {\em permutation complete} if for every word 
${\bi w}$, it holds that $\hat \A_\exists$ accepts ${\bi w}$ iff it accepts every permutation of 
${\bi w}$. 
\end{enumerate}
\end{definition}

An $\nfhf$ $\A_\forall$ accepts a hyperword $S$ iff $\hat\A_\forall$ accepts every sequence of size $k$ of $S$. If some sequence is missing from $\lang{\hat\A}$, then removing the rest of the sequences of $S$ from $\lang{\hat\A_\forall}$ does not affect the non-acceptance of $S$. Therefore, the underlying automata of sequence-complete $\nfhf$ only accept necessary sequences. 
Similarly, an $\nfhe$ $\A_\exists$ accepts a hyperword $S$ iff $\hat\A_\exists$ accepts some permutation $p$ of size $k$ of words in $S$. Adding the rest of the permutations of $p$ to $\lang{\hat\A_\exists}$ does not affect the acceptance of $S$. Therefore, the underlying automata of permutation-complete $\nfhe$ only reject the necessary permutations of every hyperword. 
As a conclusion, we have the following.

\begin{lemma}
\label{lem:langeq}
\begin{enumerate}
\item Let $\A_\forall$ be an $\nfhf$, and let $\A'_\forall$ be a sequence-complete $\nfhf$ over $\Sigma$ and $X$ such that for every word ${\bi w}$, the underlying NFA
$\hat{\A'_\forall}$ accepts ${\bi w}$ iff $\hat{\A_\forall}$ accepts every sequence of ${\bi w}$. Then $\hlang{\A_\forall} =\hlang{\A'_\forall}$.
\item Let $\A_\exists$ be an $\nfhe$, and let $\A'_\exists$ be a permutation-complete $\nfhe$ over $\Sigma$ and $X$ such that for every word ${\bi w}$, the underlying NFA $\hat{\A_\exists}$ accepts ${\bi w}$ iff $\hat{\A'_\exists}$ accepts all permutations of ${\bi w}$. 
Then $\hlang{\A_\exists} =\hlang{\A'_\exists}$.
\end{enumerate}

\end{lemma}

Next, we show that we can construct a sequence- or permutation-complete NFH for a given $\nfhf$ or $\nfhe$, respectively. Intuitively, given $\A$,
for every sequence (permutation) $\zeta$ of $(1,\ldots k)$, we construct an NFA that runs on ${\bi w}_\zeta$ in the same way that $\hat\A$ runs on $\bi w$, for every $\bi w$.
The underlying NFA we construct for the $\nfhf$ and $\nfhe$ are the intersection and union, respectively, of all these NFA. 

\begin{lemma}\label{lem:permutation.sequence.complete}
Every $\nfhf$ ($\nfhe$) $\A$ has an equivalent sequence-complete (permutation-complete) $\nfhf$ ($\nfhe$) $\A'$ over the same set of variables. 
\end{lemma}

Finally, as the following theorem shows, sequence- and permutation- complete NFH offer a unified model for the alternation-free fragments.

\begin{theorem}\label{thm:permutation.sequence.complete}
Let $\A_1$ and $\A_2$ be two sequence-complete (permutation-complete) $\nfhf$ ($\nfhe$) over the same set of variables. Then $\hlang{\A_1}=\hlang{\A_2}$ iff  $\lang{\hat\A_1} = \lang{\hat \A_2}$.
\end{theorem}

\stam{
\begin{lemma}
\label{lem:eqseq}
Every $\nfhf$ $\A$ has an equivalent sequence-complete $\nfhf$ $\A'$ over the 
same set of variables. 
\end{lemma}

\begin{lemma}
\label{lem:seqcomp}
Let $\A_1$ and $\A_2$ be two sequence-complete $\nfhf$ over the same set of 
variables $X$.
Then $\lang{\A_1}=\lang{\A_2}$ iff  $\lang{\hat\A_1} = \lang{\hat \A_2}$.
\end{lemma}
}

Regular languages have a canonical form, which are minimal DFA. We use this property to define canonical forms for $\nfhf$ and $\nfhe$ as sequence-complete (permutation-complete) $\nfhf$ ($\nfhe$) with a minimal number of variables and a minimal underlying DFA. 

\stam{

\subsubsection{A Canonical Form for $\nfhe$}

We say that an $\nfhe$ $\A'$ is {\em permutation complete} if for every word 
$w$, it holds that $\hat \A'$ accepts $w$ iff it accepts every permutation of 
$w$. 

\begin{lemma}
\label{lem:premcomp}
Let $\A$ be an $\nfhe$.
Let $\A'$ be a permutation-complete $\nfhe$ over $\Sigma$ and $X$ with the following property: for every word $w$, the underlying NFA $\hat\A$ accepts a word $w$ iff $\hat\A'$ accepts all permutations of $w$. %Notice that this is the dual property of the one listed for $\nfhf$.
Then $\lang{\A} =\lang{\A'}$.
\end{lemma}

\begin{lemma}
\label{lem:eqpermcomp}
Every $\nfhe$ has an equivalent permutation-complete $\nfhe$ over the same set 
of variables. 
\end{lemma}

\begin{lemma}
\label{lem:permcompsame}
Let $\A_1$ and $\A_2$ be two permutation-complete $\nfhe$ over the same set of 
variables $X$.
Then $\lang{\A_1}=\lang{\A_2}$ iff  $\lang{\hat\A_1} = \lang{\hat \A_2}$.
\end{lemma}

We define a canonical form for $\nfhe$ as a minimal deterministic 
permutation-complete $\nfhe$ with a minimal number of variables.


}% of stam






