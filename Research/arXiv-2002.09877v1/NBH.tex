






We now lift the notion of hyperautomata to infinite words. 
A {\em hyper-infinite word} over $\Sigma$ is a set of infinite words over $\Sigma$. 
As before, an automaton $\A$ for a hyperlanguage $\cal L$ consists of an underlying automaton which runs on the tuples of words that are assigned to the word variables of $\A$. In the case of hyper-infinite words, the underlying automaton runs on tuples of infinite words. We consider a B{\"u}chi acceptance condition for the underlying automata. 

\begin{definition}
A {\em nondeterministic B{\"u}chi hyper-word automaton (NBH)} is a tuple $\A = 
\tuple{\Sigma, X=\{x_1,\ldots x_k\}, Q, Q_0, \delta, F, \alpha}$, where all 
elements are as in NFH and acceptance condition defined by the underlying NBW 
of $\A$ is $\hat{\A} = \tuple{\Sigma^k,Q,Q_0,\delta,F}$.

\end{definition}

In particular, consider a hyper-infinite word $S$ and an 
assignment $v:X\rightarrow S$. Then $\hat\A$ accepts $\zip(v)$ iff the run of 
$\hat\A$ on $\zip(v)$ visits some state in $F$ infinitely often. 
The definition of $\models$, acceptance, and $\lang{A}$ are all as in NFH.
As with NFH, we consider the different fragments of NBH$_\forall$, NBH$_\exists$, NBH$_{\forall\exists}$ and NBH$_{\exists\forall}$.


\subsection{Properties of NBH}

Since NBWs are closed under the Boolean operations, and their nonemptiness and inclusion problems are decidable (and are NL-complete and PSPACE-complete, respectively), the results of Section~\ref{sec:nfh_properties} can all be lifted to the case of hyper-infinite words. 
To summarize, we have the following. 

\begin{theorem}
\begin{itemize}
    \item NBH are closed under the Boolean operations.
    \item The nonemptiness problem for NBH$_\exists$ is NL-complete.
    \item The nonemptiness problem for NBH$_{\exists\forall}$ is decidable in EXPSPACE.
    \item The nonemptiness problem for NFH$_{\forall\exists}$ is undecidable.
    \item The $\omega$-regular membership problem for NBH is decidable.\todo{this is just a bit trickier, since we need to be more careful with the accepting states}
\end{itemize}
\end{theorem}

