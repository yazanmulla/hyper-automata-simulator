
\begin{abstract}

{\em Hyperproperties} lift conventional trace properties from a set of 
execution traces to a set of sets of execution traces. Hyperproperties have 
been shown to be a powerful formalism for expressing and reasoning about 
information-flow security policies and important properties of cyber-physical 
systems such as sensitivity and robustness, as well as consistency conditions in 
distributed computing such as linearizability. Although there is an extensive 
body of work on automata-based representation of trace properties, we currently 
lack such characterization for hyperproperties.

We introduce {\em hyperautomata} for {\em hyperlanguages}, which are languages 
over sets of words. Essentially, hyperautomata allow running multiple quantified 
words over an automaton. We propose a specific type of hyperautomata called 
{\em nondeterministic finite hyperautomata} (NFH), which accept {\em regular 
hyperlanguages}. We demonstrate the ability of regular hyperlanguages to express 
hyperproperties for finite traces. We then explore the fundamental properties of 
NFH and show their closure under the Boolean operations.
We show that while nonemptiness is undecidable in general, it is decidable for several fragments of NFH. We further show the decidability of the membership problem for finite sets and regular languages for NFH, as well as the containment problem for several fragments of NFH. 
Finally, we introduce learning algorithms 
based on Angluin’s $\lstar$ algorithm for the fragments NFH in which the 
quantification is either strictly universal or strictly existential.

\end{abstract}