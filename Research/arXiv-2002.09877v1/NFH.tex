\subsection{Nondeterminsitic Finite-Word Hyperautomata}
\label{sec:haf}

%Here, we consider hyperwords consisting of words over an alphabet $\Sigma$.
We begin with some terms and notations. 

Let $s = (w_1,w_2,\ldots, w_k)$ be a tuple of finite words over  $\Sigma$. We 
denote the length of the longest word in $s$ by $\ceil{s}$. We represent $s$ by 
a word over $(\Sigma\cup\{\#\})^k$ of length $\ceil{s}$, which is 
formed by a function $\zip(s)$ that ``zips'' the words in $s$ together: the 
$i$'th letter in $\zip(s)$ represents the $i$'th letters in $w_1, w_2, \ldots, 
w_k$, 
and $\#$ is used to pad the words that have ended. For example,
%
$$\zip(aab, bc, abdd) = (a, b, a)(a, c, b)(b, \#, d)(\#, \#, d).$$
%
Formally, we have $\zip(s) =  \bi{s}_1\bi{s}_2\cdots \bi{s}_{\ceil{s}}$, 
where
$\bi{s}_i[j] = w_{j_i}$ if $j\leq|w|$, and $\bi{s}_i[j] = \#$, otherwise.
%For example, for $s  = (aa, bab)$, we have $zip(s) = %(a,b)(a,a)(\#,b)$. I 
%added my example and then noticed yours!

Given a zipped word $\bi{s}$, we denote the word formed by the letters in the 
$i$'th positions in $\bi{s}$ by $\bi{s}[i]$. 
That is, $\bi{s}[i]$ is the word $\sigma_1\sigma_2\cdots \sigma_m$ formed by 
defining $\sigma_j = \bi{s}_j[i]$, for $\bi{s}_j[i]\in \Sigma$.
Notice that $\zip(s)$ is reversible, and we can define an $\unzip$ function 
as $\unzip(\bi{s}) = (\bi{s}[1],\bi{s}[2],\dots, \bi{s}[k])$. We sometimes abuse the notation, and use $\unzip(\bi{s})$ to denote $\{\bi{s}[1],\bi{s}[2],\dots, \bi{s}[k]\}$, and $\zip(S)$ to denote the zipping of the words in a finite hyperword $S$ in some arbitrary order.

\begin{definition}
\label{def:nfh}
A {\em nondeterministic finite-word hyperautomaton} (NFH) is a tuple \linebreak
$\A = \tuple{\Sigma,X,Q,Q_0,F,\delta,\alpha}$, where $\Sigma$, $Q$, $Q_0$, and $F$ are as in Definition~\ref{def:nfa}, $X=\{x_1,\dots, x_k\}$ is a finite set of {\em word variables}, 
$\delta\subseteq Q\times (\Sigma\cup \{\#\})^k \times Q$ is a transition 
relation, and $\alpha  = \quant_1 x_1\quant_2x_2\ldots \quant_nx_k$ is a {\em 
quantification condition}, where $\quant_i\in\{\forall,\exists\}$ for every 
$1\leq i\leq k$.

\end{definition}
%
In Definition~\ref{def:nfh}, the tuple $\tuple{(\Sigma\cup \{\#\})^k, Q, 
Q_0, \delta, F}$ forms an underlying NFA of $\A$, which we denote by 
$\hat{\A}$. We denote the alphabet of $\hat\A$ by $\hat{\Sigma}$. 

Let $S$ be a hyperword and let $v: X\rightarrow S$ be an assignment of the word 
variables of $\A$ to words in $S$. We denote by  $v[x\rightarrow w]$ the assignment obtained from $v$ by assigning the word $w\in S$ to $x\in X$. We represent $v$ by the word $\zip(v) = \zip(v(x_1),\ldots v(x_k))$.  
%
We now define the acceptance condition of a hyperword $S$ by an NFH $\A$. We 
first define the satisfaction relation $\models$ for $S$, $\A$, a quantification condition $\alpha$, and an assignment $v:X\rightarrow S$, as follows.

\begin{itemize}
    \item For $\alpha = \epsilon$, we denote $S \models _v (\alpha,\A)$ if
$\hat\A$ accepts $\zip(v)$. 

\item For $\alpha = \exists x_i \alpha'$, we denote $S\models_v (\alpha,\A)$ if 
there exists $w\in S$, such that $S \models_{v[x_i\rightarrow w]}  
(\alpha',\A)$.

\item For $\alpha = \forall x_i \alpha'$, we denote $S\models_v (\alpha,\A)$ 
if for every $w\in S$, it holds that $S \models_{v[x_i\rightarrow w]}  
(\alpha',\A)$.\footnote{In case that $\alpha$ begins with 
$\forall$, satisfaction holds vacuously with an empty hyperword. We 
restrict the discussion to nonempty hyperwords.}

\end{itemize}
%
Since the quantification condition of $\A$ includes all of $X$, the satisfaction is independent of the 
assignment $v$, and we denote $S \models \A$, in which case, we say that {\em $\A$ 
accepts $S$}.

 

\begin{definition}
Let $\A$ be an NFH. The {\em hyperlanguage} of $\A$, denoted $\hlang{\A}$, is 
the set of all hyperwords that $\A$ accepts.
\end{definition}

We call a hyperlanguage $\hl$ a {\em regular hyperlanguage} if there exists an NFH $\A$ such that $\hlang{\A} = \hl$.

%\noindent{\bf Examples}



\begin{example}
Consider the NFH $\A_3$ in Figure~\ref{fig:ordered}, over the alphabet $\Sigma = 
\{a,b\}$ and two word variables $x_1$ and $x_2$. From the initial state, two 
words lead to the left component in $\hat{\A_3}$ iff in every position, if the 
word assigned to $x_2$ has an $a$, the word assigned to $x_1$ has an $a$. In 
the right component, the situation is dual -- in every position, if the word 
assigned to $x_1$ has an $a$, the word assigned to $x_2$ has an $a$. 
Since the quantification condition of $\A_3$ is $\forall x_1\forall x_2$, in a hyperword $S$ accepted by $\A_3$, in every two words in $S$, the set of $a$ positions of one is a subset of the $a$ positions of the other. Therefore, $\hlang{\A_3}$ includes all hyperwords in which there is a full ordering on the $a$ positions. 


\begin{figure}[ht]
%\hrulefill
    \begin{center}
        \includegraphics[scale=0.5]{figures/a_implies_a_2.pdf}
    \end{center}
    \caption{The NFH $\A_3$.}
    \label{fig:ordered}
%    \hrulefill
\end{figure}
\end{example}

We consider several fragments of NFH, which limit the structure of the quantification condition $\alpha$.
$\nfhf$ is the fragment in which $\alpha$ contains only $\forall$ quantifiers, 
and similarly, in $\nfhe$, $\alpha$ contains only $\exists$ quantifiers. In 
the fragment $\nfhef$, $\alpha$ is of the form $\exists x_1 \cdots \exists x_i \forall 
x_{i+1}\cdots \forall x_k$.

%and finally, in $\nfhfe$, $\alpha$ is of the form  
%$\forall x_1 \cdots \forall x_i \exists x_{i+1}\cdots \exists x_k$.

\subsection{Additional Terms and Notations}

We present several more terms and notations which we use throughout the following sections.
%
We say that a word $\bi w$ over $(\Sigma\cup \#)^k$ is {\em legal} if 
${\bi w}=\zip(u_1,\ldots u_k)$ for some $u_1,u_2,\ldots u_k \in \Sigma^*$. 
Note that $\bi w$ is legal iff there is no ${\bi w}[i]$ in which there is an occurrence of $\#$ 
followed by some letter $\sigma\in \Sigma$. 

Consider two letter tuples $\sigma_1 = (t_1,\ldots t_k)$ and $\sigma_2 = (s_1,\ldots s_{k'})$. We denote by $\sigma_1+\sigma_2$ the tuple $(t_1,\ldots t_k, s_1,\ldots s_{k'})$.
We extend the notion to zipped words. Let ${\bi w_1} = \zip(u_1,\ldots u_k)$ and ${\bi w_2} = \zip(v_1,\ldots v_{k'})$. We denote by ${\bi w_1} +{\bi w_2}$ the word $\zip(u_1,\ldots u_k,v_1,\ldots v_{k'})$.

Consider a tuple $t = (t_1,t_2,\ldots t_k)$ of items. 
A {\em sequence} of $t$ is a tuple $(t'_1, t'_2,\ldots t'_k)$, where 
$t'_i\in\{t_1,\ldots t_k\}$ for every $1\leq i \leq k$. A {\em permutation} of $t$ is a reordering of the elements of $t$. 
We extend these notions to zipped words, to 
assignments, and to hyperwords, as follows. Let $\zeta = 
(i_1,i_2,\ldots i_k)$ be a sequence (permutation) of $(1,2,\ldots, k)$.
\begin{itemize}
    \item Let ${\bi w} = \zip(w_1,\ldots w_k)$ be a word over $k$-tuples. The word ${\bi w}_\zeta$, 
defined as $\zip(w_{i_1}, w_{i_2}, \ldots w_{i_k})$ is a sequence (permutation) of ${\bi w}$.
    \item Let $v$ be an assignment from a set of variables $\{x_1,x_2,\ldots 
x_k\}$ to a hyperword $S$. The assignment $v_\zeta$, defined as $v_\zeta(x_j) = 
v(x_{i_j})$ for every $1\leq i,j \leq k$, is a sequence (permutation) of $v$. 
\item Let $S$ be a hyperword. The tuple ${\bi w} = (w_1,\ldots w_k)$, where $w_i\in S$, is a sequence of $S$. if $\{w_1,\ldots w_k\} = S$, then $\bi w$ is a permutation of $S$. 
\end{itemize}
