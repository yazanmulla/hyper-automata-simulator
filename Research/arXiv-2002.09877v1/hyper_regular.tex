\section{Hyperregular expressions}
\todo{Sarai: I think that we should drop the definition. We can say: We can represent a hyper-regular language $HL$ as a hyper-regular expression $\alpha r$, where $r$ is a regular expression for $\hat\A$ in an NFH $\A$ for $HL$, whose quantification condition is $\alpha$. I don't think we need more than that for this idea.}
A {\em hyper regular expression} (HRE) defines a hyperlanguage. HREs are 
inductively defined by the following grammar:
%
\todo{Sarai: there should also be $\emptyset$ in $\psi$, right?}
\begin{align*}
\varphi ::= & \ \forall x.\varphi \mid \exists x.\varphi \mid \psi\\
\psi ::= & \ \epsilon \mid (a_1, a_2, \ldots a_k) \mid \psi + \psi \mid 
\psi^* \mid \psi\psi
\end{align*}
% 
\todo{Sarai: since we use $k$-tuples, we should probably say that the set of HR is the set of closed (with $k$ quantifiers) expressions. Otherwise the semantics does not make sense.}
where $a_1, a_2, \dots, a_k$ are letters in a finite alphabet $\Sigma$ and $x$ 
is a word variable from an infinite supply $X$.

\todo{Sarai: the syntax and semantics should be separated. Regular expressions are the syntax. For every RE $r$, we assign a hyperlanguage $HL[r]$}The semantics of HREs is as follows. The following constants are 
defined as HREs (obtained by the $\psi$ rule): (1) the empty set $\emptyset$, 
(2) $\{\epsilon\}$ is an HRE, where $\epsilon$ is the empty word, and (3) a 
$k$-tuple of letters $(a_1, a_2, \ldots, a_k)$ is an HRE.
%
Furthermore, HREs can be obtained by the $\psi$ formulas as 
follows: 

\begin{itemize}
\item {\em Concatenation.} \ Let $R_1$ and $R_2$ be two sets of hyperwords. 
$R_1R_2$ denotes the set of hyperwords obtained by concatenating a hyperword in 
$R_1$ and a hyperword in $R_2$. We use the same padding technique for words 
with unequal lengths as described in Section~\ref{sec:ha}. Thus,
%
$$
R_1R_2 = \zip \Big(\unzip(R_1)\cdot\unzip(R_2)\Big)
$$
%
where `$\cdot$' denotes the the standard concatenation operator in regular 
expressions.
For example, let $R_1 
= \{(a, b, c)\}$, and $R_2 = \{(e_1, f_1)(\#, f_2)\}$. Then, 
$R_1R_2 = \{(a, a, b, b, c, c)(e_1, f_1, e_1, f_1)(\#, f_2, \#, 
f_2, \#, f_2)\}$.\todo{Sarai: I don't understand... why is this the concatenation?}

\item {\em Union.} \ $R_1 + R_2$ denotes the set union of sets described 
by $R_1$ and $R_2$:
%
$$
R_1 + R_2 = \zip\Big(\unzip(R_1) \, \cup \unzip(R_2)\Big)
$$
%
For example, if $R_1 = \{(a_1, b_1, a_2, b_2)\}$, and $R_2 
= \{(e_1, f_1)(e_2, f_2)\}$, expression $R_1 + R_2 = 
\{(a_1,b_1, a_2, b_2), (e_1, f_1 \#, \#)(e_2, f_2 \#, \#)\}$.

\item {\em Kleene star.} \ $R^*$ denotes the smallest superset of the set 
described by $R$ that contains $\epsilon$ and is closed under concatenation. 
This is the set of all hyperwords that can be made by concatenating any finite 
number (including zero) of hyperwords from the set described by $R$. For 
example, $((a, a) + (\bar{a}, \bar{a}))^*$, where $\bar{a} \in \Sigma - 
\{a\}$, is the hyperlanguage in which every pair of words agree on the 
position of letter $a$.

\end{itemize}
%
Finally, if $R$ is an HRE obtained by concatenation, alternation, and Kleen 
star only, then $\quant_1x_1.\quant_2x_2.\ldots 
\quant_nx_n R$, where $\quant_i \in \{\forall, \exists\}$, for all $i \in [1, 
n]$ is also an HRE. The semantics of quantified $\varphi$ formulas is the same 
as NHF described in Section~\ref{sec:haf}. 

\begin{theorem}

There is a direct one-to-one correspondence between the grammar of HRE and 
NHF, such that the grammar generates exactly the language the automaton accepts.

\end{theorem}

\begin{proof}
The construction is the same as the standard construction of an NFA from 
a regular expression. The quantification condition $\alpha$ of the NHF will be 
identical to the quantifiers of the input HRE. 
\end{proof}



% We clarify that an HRE cannot be obtained by (1) concatenating two quantified 
% expressions, and (2) applying Kleene star over a quantified expression. Also, 
% union should not created nested quantification.
%\subsection{Semantics of NFH}
% We now formally describe the semantics of NFH, and begin with some terms and 
% notations.


% The semantics of quantification is as follows. As we have noted, a hyperword 
% over $\Sigma$ is $S\subseteq \Sigma^*$. Let $X=\{x_1,\dots, x_k\}$ is a finite 
% set of {\em word variables} be a set of {\em word variables} and $v: 
% X\rightarrow S$ be an assignment of the word variables to words in $S$.
% We represent $v$ by the word $\zip(v) = \bi{v} = \zip(v(x_1),\ldots v(x_k))$.  
% 
% We now define the acceptance condition of a hyperword $S$ by an NFH $\A$. We 
% first define the satisfaction relation $\models$ for a hyperword $S$, an NFH 
% $\A$ over the set of variables $X = \{x_1,\ldots x_k\}$, a quantification 
% condition $\alpha$, and an assignment $v:X\rightarrow S$.
% 
% \begin{itemize}
%     \item For $\alpha = \epsilon$, we denote $S \models _v (\alpha,\A)$ iff 
% $\hat\A$ accepts $\zip(v)$. 
% 
% \item For $\alpha = \exists x_i \alpha'$, we denote $S\models_v \alpha,\A$ iff 
% there exists $w\in S$ such that $S \models_{v[x_i\rightarrow w]}  \alpha',\A$.
% 
% \item For $\alpha = \forall x_i \alpha'$, we denote $S\models_v \alpha',\A$ iff 
% for every $w\in S$, it holds that $S \models_{v[x_i\rightarrow w]}  \A$.
% 
% \end{itemize}
% When $\alpha$ includes all of $X$, then the satisfaction is independent of the 
% assignment $v$, and we denote $S \models \A$, in which case, we say that $\A$ 
% accepts $S$.



