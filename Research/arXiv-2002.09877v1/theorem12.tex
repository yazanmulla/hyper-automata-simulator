\begin{theorem}
Let $k'$ be the number of variables under $\forall$.
\begin{itemize}
    \item If $k'= O(\log(k))$ then the membership problem is NP-complete.
    \item
    If $k = O(k')$ then the membership problem is NL-complete.
\end{itemize}

\end{theorem}

\begin{proof}
Let $S$ be a hyperword of size at least $2$, and let $\A$ be an NFH. 

We can solve the membership problem for $\A$ and $S$ by iterating over relevant assignments from $X$ to $S$, and for every such assignment $v$, checking whether $\zip(v)$ is accepted by $\hat\A$. 
This algorithm uses space of size that is polynomial in $k$ and logarithmic in $|\A|$ and in $|S|$. 

To account for all the assignments to the $\forall$ variables, $\delta$ -- and therefore, $\hat\A$ -- must be of size at least $2^{k'}$ (otherwise, we can return ``no'').
We then have that if $k = O(k')$, then space of size $k$ is logarithmic in $|\hat\A|$, and so the problem in this case can be solved within logarithmic space.
The lower bound follows from the membership problem for NFA. 

In the case that $k' = O(\log k)$, iterating over 
all assignments to the variables under $\forall$, while guessing assignments to 
the variables under $\exists$ is done in polynomial time. For each such assignment $v$, checking whether $\zip(v)\in\lang{\hat\A}$ can be done on-the-fly. 
Notice that now we cannot claim that the size of $\delta$ is exponential in 
$k$. Therefore, the described solution provides membership in 
\comp{NP}. 

We show \comp{NP-hardness} for this case by a reduction from the Hamiltonian cycle problem. 
Given a graph $G = \tuple{V,E}$ where $V = \{v_1,v_2,\ldots v_n\}$ and $|E|=m$, 
we construct an $\nfhe$ $\A$ over $\{0,1\}$ with $n$ states, $n$ variables, 
$\delta$ of size $m$, and a hyperword $S$ of size $n$, as follows. $S = 
\{w_1,\ldots w_n\}$, where $w_i$ is the word over $\{0,1\}$ in which all letters 
are $0$ except for the $i$'th. 
The structure of $\hat\A$ is identical to that of $G$, and we set $Q_0 = F = 
\{v_1\}$. For the transition relation, for every $(v_i,v_j)\in E$, we have 
$(v_i, \sigma_i,v_j)\in \delta$, where $\sigma_i$ is the letter over $\{0,1\}^n$ 
in which all positions are $0$ except for position $i$. 
Intuitively, the $i$'th letter in an accepting run of $\hat\A$ marks traversing 
$v_i$. Assigning $w_j$ to $x_i$ means that the $j$'th step of the run 
traverses $v_i$. Since the words in $w$ make sure that every $v\in V$ is 
traversed exactly once, and that the run on them is of length $n$, we have that 
$\A$ accepts $S$ iff there exists some permutation $p$ of the words in $S$ such 
that $p$ matches a Hamiltonian cycle in $G$.
\end{proof}