\section{Properties of Hyper Regular Languages}
\label{sec:nfh_properties}


In this section, we consider the basic operations and decision problems for the various fragments of NFH. 
Throughout  section, we assume an NFH $\A = \tuple{\Sigma,X,Q,Q_0,\delta,F,\alpha}$ where $X = \{x_1,\ldots x_n\}$. 

%For the various complexity results, notice that  the number of variables $k$ and the number of letters $\Sigma$ dictate an alphabet size of $O(|\Sigma|^k)$ for $\hat\A$. Thus, though the state space of $\hat\A$ is of size $n$, its size may be exponential in the number of variables due to its transition relation, which may be of size $n\times |\Sigma|^k\times n$. 


\begin{theorem} 
\label{thm:nfh.complement}

NFH are closed under complementation.

\end{theorem}


\begin{theorem} \label{thm:nfh.union}
NFH are closed under union.
\end{theorem}

\begin{theorem}\label{thm:nfh.intersection}


NFH are closed under intersection.
\end{theorem}


We now turn to study various decision problems for NFH. We begin with the 
nonemptiness problem: given an NFH $\A$, is $\hlang{\A} = \emptyset$? We show 
that while the problem is in general undecidable for NFH, it is decidable for 
most fragments that we consider. 


\begin{theorem}
\label{thrm:nonemptiness}

The nonemptiness problem for NHF is undecidable.
\end{theorem}



\begin{theorem} \label{thm:nfhe.nfhf.nonemptiness}
The nonemptiness problem for $\nfhe$ and $\nfhf$ is \comp{NL-complete}.
\end{theorem}

We now show that the nonemptiness problem for $\nfhef$ is decidable. We first 
present some terms and notations.  

Consider a tuple $t = (t_1,t_2,\ldots t_k)$ of items. 
A {\em sequence} of $t$ is a tuple $(t'_1, t'_2,\ldots t'_k)$, where 
$t'_i\in\{t_1,\ldots t_k\}$ for every $1\leq i \leq k$. 
A {\em permutation} of $t$ is a reordering of the elements of $t$. 
We extend the notion of permutation to words over $k$-tuples, and to 
assignments, as follows. 
Let $w = \zip(w_1,\ldots w_k)$ be a word over $k$-tuples, and let $\zeta = 
(i_1,i_2,\ldots i_k)$ be a sequence (permutation) of $(1,2,\ldots, k)$. The word $w_\zeta$, 
defined as $\zip(w_{i_1}, w_{i_2}, \ldots w_{i_k})$ is a sequence (permutation) of $w$.
Finally, let $v$ be an assignment from a set of variables $\{x_1,x_2,\ldots 
x_k\}$ to a hyperword $S$. The assignment $v_\zeta$, defined as $v_\zeta(x_j) = 
v(x_{i_j})$ for every $1\leq i,j \leq k$, is a sequence (permutation) of $v$. 
\todo{Sarai: move this elsewhere? since we use these terms also in learning}

We now show that if an $\nfhef$ is nonempty, then we can bound the size of a 
hyperword that it accepts.

\begin{lemma}\label{lemma:nfhef.nonempty}
Let $\A$ be an $\nfhef$ with a quantification condition $\alpha = \exists x_1,\ldots \exists x_m \forall 
x_{m+1}\ldots \forall x_k$, where $1 \leq m < k$. 
Then $\A$ is nonempty iff $\A$ accepts a hyperword of size at most $m$.
\end{lemma}


We now use Lemma~\ref{lemma:nfhef.nonempty} to describe a decision procedure for 
the nonemptiness of $\nfhef$. 

\begin{theorem}\label{thm:nfhef.nonemptiness}
The nonemptiness problem for an $\nfhef$ $\A$ can be decided in space that is 
polynomial in the size of $\hat\A$.
\end{theorem}


\begin{theorem}
\label{thrm:nonemptinessEA}
The nonemptiness problem for $\nfhef$ is  \comp{PSPACE-complete}.
\end{theorem}

We turn to study the membership problem for NFH: given an NFH $\A$ and a 
hyperword $S$, is $S\in\hlang{\A}$? 
When $S$ is finite, the set of assignments from $X$ to $S$ is finite, and so the problem is decidable. We call this case the {\em finite membership problem}. 

\begin{theorem}\label{thm:nfh.membership.finite}
Let $\A$ be an NFH and $S$ be a finite hyperword over $\Sigma$. Then it can be 
decided whether $S\in \hlang{\A}$ in space that is polynomial in $k$, and 
logarithmic in $|\hat\A|, |S|$.
\end{theorem}

\begin{proof}
As discussed in the proof of Theorem~\ref{thm:nfhef.nonemptiness}, the size of 
$\hat\Sigma$ must be exponential in the number of $\forall$ quantifiers in 
$\alpha$, and therefore is exponential in $k$. We can decide the membership of 
$S$ in $\hlang{\A}$ by iterating over all assignments of the type $X\rightarrow 
S$. For every such assignment $v$, we construct $\zip(v)$ and run $\hat\A$ on 
$\zip(v)$, on-the-fly. 
\end{proof}

The exponential size of $\hat\A$ is derived from the number of $\forall$ 
quantifiers. When the number of $\forall$ quantifiers is fixed, then 
$\hat\Sigma$ is not necessarily exponential in $k$. 

\begin{theorem}
\label{thrm:membershipA}
The finite membership problem for NFH with a fixed number of $\forall$ 
quantifiers is \comp{NP-complete}.
\end{theorem}

When $S$ is infinite, it may still be finitely represented. 
We now address the problem of deciding whether a regular language $\cal L$ 
(given as an NFA) is accepted by an NFH. We call this {\em the regular 
membership problem for NFH}. We show that this problem is decidable for the 
entire class of NFH. 

\begin{theorem}
\label{thrm:membershipFULL}
The regular membership problem for NFH is decidable.
\end{theorem}


\begin{theorem}
\label{thrm:inclusion}
 The inclusion problems of $\nfhe$ and $\nfhf$ in $\nfhe$ and $\nfhf$ and of 
$\nfhef$ in $\nfhe$ and $\nfhf$ are \comp{PSPACE-complete}.
\end{theorem}

\iffalse

\subsection{Properties of NBH}

Since NBWs are closed under the Boolean operations, and their nonemptiness and inclusion problems are decidable (and are NL-complete and PSPACE-complete, respectively), the results of Section~\ref{sec:nfh_properties} can all be lifted to the case of hyper-infinite words. 
To summarize, we have the following. 

\begin{theorem}
\begin{itemize}
    \item NBH are closed under the Boolean operations.
    \item The nonemptiness problem for NBH$_\exists$ is NL-complete.
    \item The nonemptiness problem for NBH$_{\exists\forall}$ is decidable in EXPSPACE.
    \item The nonemptiness problem for NFH$_{\forall\exists}$ is undecidable.
    \item The $\omega$-regular membership problem for NBH is decidable.\todo{this is just a bit trickier, since we need to be more careful with the accepting states}
\end{itemize}
\end{theorem}

\fi