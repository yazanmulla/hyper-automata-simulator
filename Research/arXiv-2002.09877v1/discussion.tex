\section{Conclusion and Future Work}
\label{sec:concl}

We have introduced and studied {\em hyperautomata} and {\em hyperlanguages}, 
focusing on 
the basic model of regular hyperlanguages, in which the underlying automaton is 
a standard NFA. We have shown that regular hyperlanguages are closed under 
set operations (complementation, intersection, and union) and are capable of 
expressing important hyperproperties for information-flow security policies over 
finite traces. We have also investigated fundamental decision procedures such as 
checking nonemptiness and membership. We have shown that their regular 
properties allow the learnability of the alternation-free fragments. Fragments 
that combine the two types of quantifiers prove to be more challenging, and we 
leave their learnability to future work.

The notion of hyperlanguages, as well as the model of hyperautomata, can be 
lifted to handle hyperwords that consist of {\em infinite} words: instead 
of an underlying finite automaton, we can use any model that accepts infinite 
words. In fact, we believe using an underlying alternating B{\"u}chi automaton, 
such hyperautomata can express the entire logic of HyperLTL~\cite{cfkmrs14}, 
using the standard Vardi-Wolper construction for LTL~\cite{VW94} as 
basis. Our complexity results for the various decision procedures for NFH, 
combined with the complexity results shown in~\cite{fh16}, suggest that using 
hyperautomata would be optimal, complexity-wise, for handling HyperLTL.

Further future directions include studying non-regular hyperlanguages 
(e.g., context-free), and object hyperlanguages (e.g., trees). Other open 
problems include a full investigation of the complexity of decision procedures 
for alternating fragments of NFH. 


